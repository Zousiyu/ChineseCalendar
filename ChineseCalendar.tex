% !TEX program=lualatex

% nobabel means you'll need to provide your own month names, week day headings,
% etc
\documentclass[12pt,nobabel,sundayweek]{cdcalendar}

%% and here we do some settings for a calendar in Chinese
%% (*not* the lunar calendar! Just localising the Gregorian calendar into
%% Chinese)
\usepackage{zh-mod}

\setmainfont[Ligatures=TeX]{EB Garamond}
\setsansfont[Ligatures=TeX,BoldItalicFont=Fira Sans Italic,BoldFont=Fira Sans,Numbers=OldStyle]{Fira Sans Light}

\title{Desktop Calendar (fits CD jewel case) with Chinese Localisation}

\usepackage{graphicx}
\usepackage{fontawesome}
\usepackage{wallpaper}
\usepackage{fontspec}

%\newfontfamily\gudian{WenYue-GuDianMingChaoTi-NC-W5.otf}


\graphicspath{{img/}}

%% Define all event mark styles here
\tikzset{holiday/.style={rectangle,fill=orange!70}}
\tikzset{pink icon/.style={text=Pink,font=\large}}
\tikzset{blue icon/.style={text=SkyBlue,font=\large}}

\begin{document}

%%%%%%
% Cover
%%%%%%

% Remove this line if you feel the background pattern is too annoying
% \TileWallPaper{.5\paperwidth}{.5\paperheight}{ricepaper_v3}
\TileWallPaper{.5\paperwidth}{.1\paperheight}{lightpaperfibers}

% \coverBgColor{RoyalBlue!40!black}
\coverImage[\large{丙申年腊月初五,思宇}]
{new_year}

\coverTitle[%
font=\fontsize{30pt}{32pt}\kaishu\bfseries,
text width=\linewidth,align=flush right,red!50!RedViolet]
{二〇一七年历}

\makeCover
% 由于须要以LuaLaTeX编译,封面变换了字体大小以后,会影响到后面英汉字符之间的距离,只好再强行加这一行来修正。
\ltjsetparameter{xkanjiskip=.25\zw plus 1pt minus 1pt}

%%%% Remove this page if you don't need it --
%%%% just to show the actual output
%{\centering
%    \includegraphics[width=\textwidth]{actual}
%    \par}
\vspace*{\fill}

\begin{center}
    {\fontsize{40pt}{\baselineskip}\kaishu 喜看阳春花千树
        \vspace{3cm}
    笑饮新岁酒一杯
    \par}
\end{center}

\vspace*{\fill}

\clearpage
%%%%

% You may find the gap between illustrations and events too wide
% Use this length to lessen it
\setlength{\lessIllusSkip}{1.5\ccwd}


%%%%%%
% Some settings for the monthly calendars
%%%%%%
\dayHeadingStyle{font=\sffamily\color{gray!90}}
\sundayColor{red}
\monthTitleStyle{font={\fontsize{42pt}{44pt}\bfseries\sffamily\fangsong}, red!50!RedViolet}
\eventStyle{\scriptsize\songti}
\renewcommand\printeventname[1]{{\heiti #1}}

\illustration[人物双鸡,林墉,当代画家。]
{8.5cm}{Lin_yong}

\begin{monthCalendar}{2017}{01}

\event[mark style=holiday]{2017-01-05}{}{腊八节}
%%\event[mark style=holiday]{2017-01-01}{2}{元旦+补休}
\event[mark style=holiday]{2017-01-27}{2017-2-2}{春节连休}
%\event[mark style=blue icon, marker=\faBriefcase]
%{2017-01-22}{}{春节调休上班}

\end{monthCalendar}

\clearpage

% 这一页"图"和月历距离可以大一些
%\setlength{\lessIllusSkip}{0pt}
%\otherstuff[金玉良言,苦口良药]
%{8.5cm}{{\fontsize{48pt}{52pt}\fangsong\textcolor{red}{写}}%
%        {\LARGE\kaishu,不停地写。}}

\illustration[柳。金农,清代书画家,扬州八怪之首。字寿门、司农、吉金,号冬心先生、稽留山民、曲江外史、昔耶居士……]
{8.5cm}{Jin_Nong_Willow}

\begin{monthCalendar}{2017}{02}

\event[mark style=holiday]{2017-01-27}{2017-02-02}{春节连休}
\event[mark style=holiday]{2017-02-11}{}{元宵节}
%\event[mark style=blue icon, marker=\faBriefcase]
%{2017-02-04}{}{春节调休上班}
%\event[mark style=pink icon,marker=\faBirthdayCake]{2017-02-14}{}{朋友生日}
%\event{2017-02-22}{}{稿件死线!!}

\end{monthCalendar}

\clearpage

\illustration[花。金农,清代书画家,扬州八怪之首。字寿门、司农、吉金,号冬心先生、稽留山民、曲江外史、昔耶居士……]
{8.5cm}{Flower_JinNong}

\begin{monthCalendar}{2017}{03}
    
\event{2017-03-05}{}{惊蛰}
\event[mark style=holiday]{2017-03-08}{}{妇女节}
\event{2017-03-20}{}{春分}

\end{monthCalendar}

\clearpage

\illustration[借问酒家何处有?牧童遥指杏花村。]
{8.5cm}{Qingming}

\begin{monthCalendar}{2017}{04}

\event{2017-04-04}{}{清明}
\event{2017-04-20}{}{谷雨}
\event[mark style=holiday]{2017-04-29}{2017-05-01}{五一小长假}

\end{monthCalendar}

\clearpage

%\illustration[借问酒家何处有?牧童遥指杏花村。]
%{8.5cm}{Qingming}

\illustration[牡丹富贵。陆抑非,名翀,自号非翁,又名苏叟。]
{8.5cm}{Lu_yifei_mudanfugui}

\begin{monthCalendar}{2017}{05}

\event[mark style=holiday]{2017-04-29}{2017-05-01}{五一小长假}
\event{2017-05-05}{}{立夏}
\event{2017-05-21}{}{小满}
\event[mark style=holiday]{2017-05-28}{2017-05-30}{端午小长假}

\end{monthCalendar}

\clearpage

\illustration[水牛。黄胄,当代画家,名淦堂,字映斋。]
{8.5cm}{Huang_zhou_shuiniu}

\begin{monthCalendar}{2017}{06}
    
\event{2017-06-05}{}{芒种}
\event{2017-06-21}{}{夏至}

\end{monthCalendar}

\clearpage

\illustration[虾。张大壮,当代画家,原名颐,又名心源,字养初,号养卢,别署富春山人。]
{8.5cm}{Zhang_dazhuang_xia}

\begin{monthCalendar}{2017}{07}

\event{2017-07-07}{}{小暑}
\event{2017-07-22}{}{大暑}

\end{monthCalendar}

\clearpage

\illustration[松鸟。黎雄才,当代画家。]
{8.5cm}{Li_xiongcai_songniao}

\begin{monthCalendar}{2017}{08}

\event{2017-08-07}{}{立秋}
\event{2017-08-23}{}{处暑}
\event[mark style=holiday]{2017-08-28}{}{七夕}

\end{monthCalendar}

\clearpage

\illustration[三峡行舟。黄君璧,当代画家,原名允瑄,本名韫之,号君璧。]
{8.5cm}{Huang_junbi_sanxiaxingzhou}

\begin{monthCalendar}{2017}{09}

\event{2017-09-07}{}{白露}
\event[mark style=holiday]{2017-09-10}{}{教师节快乐}
\event{2017-09-23}{}{秋分}

\end{monthCalendar}

\clearpage

\illustration[菊。黎雄才,当代画家。]
{8.5cm}{Li_xiongcai_ju}

\begin{monthCalendar}{2017}{10}

\event[mark style=holiday]{2017-10-01}{2017-10-08}{举国同庆}
\event[mark style=holiday]{2017-10-04}{}{中秋节}
\event{2017-10-08}{}{寒露}
\event{2017-10-23}{}{霜降}
\event[mark style=holiday]{2017-10-28}{}{重阳}
\end{monthCalendar}

\clearpage

\illustration[独钓寒江雪。赵少昂,当代画家,字叔仪。]
{8.5cm}{Zhao_shaoang_dudiaohanjiangxue}

\begin{monthCalendar}{2017}{11}
    
\event{2017-11-07}{}{立冬}
\event{2017-11-22}{}{小雪}

\end{monthCalendar}

\clearpage

\illustration[洪荒风雪。黄胄,当代画家,名淦堂,字映斋。]
{8.5cm}{Huang_zhou_honghuangfengxue}

\begin{monthCalendar}{2017}{12}

\event{2017-12-07}{}{大雪}
\event{2017-12-22}{}{冬至}
\event[mark style=holiday]{2017-12-25}{}{圣诞}

\end{monthCalendar}

\end{document}
